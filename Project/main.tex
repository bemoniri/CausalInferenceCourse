\documentclass{article}
%----------------------------------------------------------------------------------------
%	PACKAGES AND OTHER DOCUMENT CONFIGURATIONS
%----------------------------------------------------------------------------------------

\usepackage{amsmath,amsfonts,stmaryrd,amssymb} % Math packages
\usepackage{algorithm}
\usepackage{algorithmic}
\usepackage{enumerate} % Custom item numbers for enumerations
\usepackage{tikz}
\usetikzlibrary{arrows}
%\usepackage[ruled]{algorithm2e} % Algorithms

\usepackage[framemethod=tikz]{mdframed} % Allows defining custom boxed/framed environments

\usepackage{listings} % File listings, with syntax highlighting
\lstset{
	basicstyle=\ttfamily, % Typeset listings in monospace font
}

%----------------------------------------------------------------------------------------
%	DOCUMENT MARGINS
%----------------------------------------------------------------------------------------

\usepackage{geometry} % Required for adjusting page dimensions and margins
\usepackage{graphicx, amsmath, amsthm, amssymb, listings, multirow, wrapfig}
\newcommand{\bigCI}{\mathrel{\text{\scalebox{1.07}{$\perp\mkern-10mu\perp$}}}}
\usepackage{xepersian}
\settextfont[Scale=1.2]{XB Niloofar}
\setdigitfont[Scale=1.2]{Yas}

\geometry{
	paper=a4paper, % Paper size, change to letterpaper for US letter size
	top=2.5cm, % Top margin
	bottom=3cm, % Bottom margin
	left=2.5cm, % Left margin
	right=2.5cm, % Right margin
	headheight=14pt, % Header height
	footskip=1.5cm, % Space from the bottom margin to the baseline of the footer
	headsep=1.2cm, % Space from the top margin to the baseline of the header
	%showframe, % Uncomment to show how the type block is set on the page
}

%----------------------------------------------------------------------------------------
%	COMMAND LINE ENVIRONMENT
%----------------------------------------------------------------------------------------

% Usage:
% \begin{commandline}
%	\begin{verbatim}
%		$ ls
%		
%		Applications	Desktop	...
%	\end{verbatim}
% \end{commandline}

\mdfdefinestyle{commandline}{
	leftmargin=10pt,
	rightmargin=10pt,
	innerleftmargin=15pt,
	middlelinecolor=black!50!white,
	middlelinewidth=2pt,
	frametitlerule=false,
	backgroundcolor=black!5!white,
	frametitle={Command Line},
	frametitlefont={\normalfont\sffamily\color{white}\hspace{-1em}},
	frametitlebackgroundcolor=black!50!white,
	nobreak,
}

% Define a custom environment for command-line snapshots
\newenvironment{commandline}{
	\medskip
	\begin{mdframed}[style=commandline]
}{
	\end{mdframed}
	\medskip
}

%----------------------------------------------------------------------------------------
%	FILE CONTENTS ENVIRONMENT
%----------------------------------------------------------------------------------------

% Usage:
% \begin{file}[optional filename, defaults to "File"]
%	File contents, for example, with a listings environment
% \end{file}

\mdfdefinestyle{file}{
	innertopmargin=1.6\baselineskip,
	innerbottommargin=0.8\baselineskip,
	topline=false, bottomline=false,
	leftline=false, rightline=false,
	leftmargin=2cm,
	rightmargin=2cm,
	singleextra={%
		\draw[fill=black!10!white](P)++(0,-1.2em)rectangle(P-|O);
		\node[anchor=north west]
		at(P-|O){\ttfamily\mdfilename};
		%
		\def\l{3em}
		\draw(O-|P)++(-\l,0)--++(\l,\l)--(P)--(P-|O)--(O)--cycle;
		\draw(O-|P)++(-\l,0)--++(0,\l)--++(\l,0);
	},
	nobreak,
}

% Define a custom environment for file contents
\newenvironment{file}[1][File]{ % Set the default filename to "File"
	\medskip
	\newcommand{\mdfilename}{#1}
	\begin{mdframed}[style=file]
}{
	\end{mdframed}
	\medskip
}

%----------------------------------------------------------------------------------------
%	NUMBERED QUESTIONS ENVIRONMENT
%----------------------------------------------------------------------------------------

% Usage:
% \begin{question}[optional title]
%	Question contents
% \end{question}

\mdfdefinestyle{question}{
	innertopmargin=1.2\baselineskip,
	innerbottommargin=0.8\baselineskip,
	roundcorner=5pt,
	nobreak,
	singleextra={%
		\draw(P-|O)node[xshift=1em,anchor=west,fill=white,draw,rounded corners=5pt]{%
		Question \theQuestion\questionTitle};
	},
}

\newcounter{Question} % Stores the current question number that gets iterated with each new question

% Define a custom environment for numbered questions
\newenvironment{question}[1][\unskip]{
	\bigskip
	\stepcounter{Question}
	\newcommand{\questionTitle}{~#1}
	\begin{mdframed}[style=question]
}{
	\end{mdframed}
	\medskip
}

%----------------------------------------------------------------------------------------
%	WARNING TEXT ENVIRONMENT
%----------------------------------------------------------------------------------------

% Usage:
% \begin{warn}[optional title, defaults to "Warning:"]
%	Contents
% \end{warn}

\mdfdefinestyle{warning}{
	topline=false, bottomline=false,
	leftline=false, rightline=false,
	nobreak,
	singleextra={%
		\draw(P-|O)++(-0.5em,0)node(tmp1){};
		\draw(P-|O)++(0.5em,0)node(tmp2){};
		\fill[black,rotate around={45:(P-|O)}](tmp1)rectangle(tmp2);
		\node at(P-|O){\color{white}\scriptsize\bf !};
		\draw[very thick](P-|O)++(0,-1em)--(O);%--(O-|P);
	}
}

% Define a custom environment for warning text
\newenvironment{warn}[1][Warning:]{ % Set the default warning to "Warning:"
	\medskip
	\begin{mdframed}[style=warning]
		\noindent{\textbf{#1}}
}{
	\end{mdframed}
}

%----------------------------------------------------------------------------------------
%	INFORMATION ENVIRONMENT
%----------------------------------------------------------------------------------------

% Usage:
% \begin{info}[optional title, defaults to "Info:"]
% 	contents
% 	\end{info}

\mdfdefinestyle{info}{%
	topline=false, bottomline=false,
	leftline=false, rightline=false,
	nobreak,
	singleextra={%
		\fill[black](P-|O)circle[radius=0.4em];
		\node at(P-|O){\color{white}\scriptsize\bf i};
		\draw[very thick](P-|O)++(0,-0.8em)--(O);%--(O-|P);
	}
}

% Define a custom environment for information
\newenvironment{info}[1][Info:]{ % Set the default title to "Info:"
	\medskip
	\begin{mdframed}[style=info]
		\noindent{\textbf{#1}}
}{
	\end{mdframed}
}
\newtheorem{den}{{\large\bf تعریف}}[section]
\newtheorem{exa}{{\large\bf مثال}}[section]
\newtheorem{lem}{{\large\bf لم}}[section]
\newtheorem{pro}{{\large\bf گزاره}}[section]
\newtheorem{cor}{{\large\bf نتیجه}}[section]
\newtheorem{thm}{{\large\bf قضیه}}[section]
\newtheorem{rem}{{\large\bf تذکر}}[section]
\newtheorem{nnt}{{\large\bf توجه}}[section]

%\newenvironment{proof}{{\bf اثبات.} \rm }{\hfill{$\blacksquare$}}
\newcommand{\bt}{\begin{theorem}}
	\newcommand{\et}{\end{theorem}}
\newcommand{\bl}{\begin{lemma}}
	\newcommand{\el}{\end{lemma}}
\newcommand{\bc}{\begin{corollary}}
	\newcommand{\ec}{\end{corollary}}
\newcommand{\bp}{\begin{proof}}
	\newcommand{\ep}{\end{proof}}
\newcommand{\be}{\begin{example}}
	\newcommand{\ee}{\end{example}}
\newcommand{\bd}{\begin{definition}}
	\newcommand{\ed}{\end{definition}}
\newcommand{\ba}{\begin{align}}
\newcommand{\ea}{\end{align}}
\newcommand{\br}{\begin{remark}}
	\newcommand{\er}{\end{remark}}
\newcommand{\bpr}{\begin{proposition}}
	\newcommand{\epr}{\end{proposition}}
\newcommand{\no}{\nonumber}
\def\B{{\frak B}}
\def\lra{\longrightarrow}
\def\lla{\longleftarrow}
\def\FX{{\cal F}(X)}
\def\AX{{\cal A}(X)}
\def\IX{{\cal I}(X)}
\def\NX{{\cal N}(X)}
\def\BX{{\cal B}(X)}
\def\KX{{\cal K}(X)}
\def\A{{\frak A}}
\def\hat{\widehat}
\def\varepsilon{\epsilon}
\def\tr{{\rm tr}}
\def\Tr{{\rm Tr}}
\def\.{{\cdot}}
\def\la{\lambda}
 
\title{طرح اولیه‌ی پروژه‌ی درس استنتاج علّی\\
\vspace{0.5cm}
\textbf{\lr{A Review on Additive Noise Models:\\ Theorems and Learning Algorithms}}
} 
\author{بهراد منیری\\95109564\\ \texttt{bemoniri@live.com}}
\date{دانشکده‌ی مهندسی برق - دانشگاه صنعتی شریف}
%----------------------------------------------------------------------------------------

\begin{document}
\maketitle
\section{شرح مختصر طرح}
\noindent
در این پروژه قصد دارم تا  مقالات مربوط به 
\lr{Additive Noise Models}
را به طور دقیق بررسی کنم. هدف من بررسی مقالات این حوزه، از ابتدا تا الگوریتم‌های جدیدی مانند RESIT و مقایسه‌ی  عملکرد  روش‌ هرکدام از آنها بر داده‌های تجربی است. در نهایت نیز قصد دارم تا این الگوریتم‌ها را  با دیتاست‌های دیگری آزمایش کنم و  مقایسه‌ای از عملکرد آنها ارائه دهم.
 البته توجه من تنها به الگوریتم‌ها معطوف نخواهد بود.  قصد دارم که اثبات‌های مربوط به «قابل شناسایی بودن علّت در مدل‌های
\lr{ANM}»
را نیز یک بار به طور دقیق مطالعه کنم تا دید دقیق‌تری به شرایط  هر یک از این قضایا پیدا کنم.

\indent
\section{مقالات}
	مقالات زیر، مقالاتی هستند که قصد دارم در طی این پروژه به بررسی آنها بپردازم.

\underline{تمرکز من، بیشتر بر دو مقاله‌ی آخر خواهد بود.}


\begin{enumerate}
	
\begin{latin}
\bibitem{ANM}  
Hoyer, P., Janzing, D., Mooij, J., Peters, J. and Sch{\"o}lkopf, B.  (2008).
\textit{Nonlinear causal Discovery with Additive Noise Models}, \textbf{Advances  in  Neural  Information  Processing  Systems 21 (NIPS 2008)}.
\end{latin}
در این مقاله، برای دو متغیر که $SCM$ای به شکل زیر دارند اثبات می‌شود که با دیتای $observational$ می‌توان جهت درست علیّت را تشخیص داد. در واقع اثبات می کند که چگالی احتمال‌های معدودی هستند که اجازه‌ی وجود  $ANM$ در دو جهت را می‌دهند. در نهایت نیز روشی برای تشخیص این جهت ارائه شده و بر روی دیتای واقعی آزمایش می‌شود.
\begin{equation}
\begin{cases}
X = N_x \\
Y = f(X) + N_y \;\;\;\;\; N_x \bigCI N_y
\end{cases}
\end{equation}

\begin{latin}
\item Zhang, K. and Hyvärinen, A. (2009). 
\textit{On the Identifiability of the Post-Nonlinear Causal Model}, \textbf{}
\textbf{UAI '09 Proceedings of the Twenty-Fifth Conference on Uncertainty in Artificial Intelligence}, pp. 647-655.
\end{latin}
در این مقاله، کار
\lr{Hoyer et al. (2008)}
به $SCM$هایی به شکل 
\begin{equation}
\begin{cases}
X = N_x \\
Y = g(f(X) + N_y) \;\;\;\;\; N_x \bigCI N_y
\end{cases}
\end{equation}
تعمیم می‌یابد و روشی برای یافتن توابع $f$ و $g$ از دیتای مشاهداتی ارائه می‌شود. در نهایت نیز این روش بر روی دیتای واقعی آزمایش می‌شود.

\begin{latin}
\item Peters, J., Janzing, D. and Sch{\"o}lkopf B. (2010).
\textit{Identifying Cause and Effect on Discrete Data using Additive Noise Models}, 
\textbf{Proceedings of the Thirteenth International Conference on Artificial Intelligence and Statistics}, PMLR 9, pp. 597-604.
\end{latin}
این مقاله مسئله‌ای مشابه با مسئله‌ی 
\lr{Hoyer et al. (2008)}
را برای داده‌های گسسته بررسی می‌کند و نشان می‌دهد که در این حالت نیز توابع چگالی احتمال مشترک معدودی هستند که برای آنها ANM هایی در دو جهت وجود دارد. در ادامه الگوریتمی بهینه برای تشخیص علّت با دیتای محدود ارائه شده و با  داده‌های واقعی آزمایش می‌شود.

\begin{latin}
\bibitem{review}  
Peters, J. , Mooij, J., Janzing, D., and Sch{\"o}lkopf, B.  (2014).
\textit{Causal Discovery with Continuous Additive Noise Models}, \textbf{Journal of Machine Learning Research 15}, pp. 2009-2053.
\end{latin}
این مقاله در ابتدا مروری بسیار جامع بر مقالات قبل انجام می‌دهد و الگوریتمی عملی (و بهینه) به نام  $RESIT$ ارائه می‌دهد. این روش مبتنی بر 
\lr{independence score}
 است. در این الگوریتم هزینه‌ای برای افزایش تعداد یال‌ها قرار داده شده است تا در نهایت به گرافی برسیم که 
\lr{Causal Minimal}
است. 
این مقاله دو  الگوریتم دیگر را نیز بررسی می‌کند. الگوریتم اول مبتنی بر جستجوی حریسانه است. به این معنی که در هر مرحله گرافی «مجاور» با گراف مرحله‌ی حاضر را پیدا می‌کند که بیشترین score را داشته باشد. الگوریتم دوم نیز که برای گراف‌های بسیار کوچک استفاده می‌شود، بررسی تک‌تک گراف‌های ممکن و یافتن گرافی با بیشترین score است.

در انتها این مقاله بحث بسیار جامع و مفصلی در مورد بهره‌گیری این سه الگوریتم بر داده‌های واقعی انجام می‌دهد و آنها را با سایر روش‌ها مانند روش $PC$، که تنها CPDAG مربوطه را به دست می‌آورند انجام می‌دهد و در نهایت مشاهده می‌کند که در مجموع، روش RESIT بر دیگر روش‌ها برتری دارد.

\begin{latin}
	\bibitem{review}  
	Mooij, J., Peters, J.,  Janzing, D., Zscheischler, J. and Sch{\"o}lkopf, B.  (2016).
	\textit{Distinguishing Cause from Effect Using Observational Data:
		Methods and Benchmarks}, \textbf{Journal of Machine Learning Research 17}, pp. 1-102.
\end{latin}
این مقاله مجدداً به مرور ANM و IGCI می‌پردازد. تمرکز من در مطالعه‌ی این مقاله، بخش‌های مربوط به ANM خواهد بود. این مقاله،‌ الگوریتم‌های مربوط به ANM را برای بیش از صد دیتاست مختلف آزمایش می‌کند و مقایسه‌ی بسیار جامعی از آنها ارائه می‌دهد.  همچنین اثبات می‌کند که الگوریتم معرفی شده توسط 
\lr{Hoyer et al. (2008)}
سازگار است. این مقاله به دلیل بررسی بسیار بسیار جامع و عمیق مدل‌های ANM می‌تواند بسیار آموزنده باشد.
\end{enumerate}
\end{document}